%% start of file `template.tex'.
%% Copyright 2006-2015 Xavier Danaux (xdanaux@gmail.com).
%
% This work may be distributed and/or modified under the
% conditions of the LaTeX Project Public License version 1.3c,
% available at http://www.latex-project.org/lppl/.


\documentclass[11pt,a4paper,sans]{moderncv}        % possible options include font size ('10pt', '11pt' and '12pt'), paper size ('a4paper', 'letterpaper', 'a5paper', 'legalpaper', 'executivepaper' and 'landscape') and font family ('sans' and 'roman')

% moderncv themes
\moderncvstyle{casual}                             % style options are 'casual' (default), 'classic', 'banking', 'oldstyle' and 'fancy'
\moderncvcolor{blue}                               % color options 'black', 'blue' (default), 'burgundy', 'green', 'grey', 'orange', 'purple' and 'red'
%\renewcommand{\familydefault}{\sfdefault}         % to set the default font; use '\sfdefault' for the default sans serif font, '\rmdefault' for the default roman one, or any tex font name
%\nopagenumbers{}                                  % uncomment to suppress automatic page numbering for CVs longer than one page

% character encoding
\usepackage[T1]{fontenc}
\usepackage[utf8]{inputenc}                       % if you are not using xelatex ou lualatex, replace by the encoding you are using
%\usepackage{CJKutf8}                              % if you need to use CJK to typeset your resume in Chinese, Japanese or Korean

% adjust the page margins
\usepackage[scale=0.75]{geometry}
%\setlength{\hintscolumnwidth}{3cm}                % if you want to change the width of the column with the dates
%\setlength{\makecvtitlenamewidth}{10cm}           % for the 'classic' style, if you want to force the width allocated to your name and avoid line breaks. be careful though, the length is normally calculated to avoid any overlap with your personal info; use this at your own typographical risks...

% personal data
\name{Inaki}{Anduaga}
\title{Curriculum Vitae}                               % optional, remove / comment the line if not wanted
\address{street and number}{postcode city}{country}% optional, remove / comment the line if not wanted; the "postcode city" and "country" arguments can be omitted or provided empty
\phone[mobile]{+1~(234)~567~890}                   % optional, remove / comment the line if not wanted; the optional "type" of the phone can be "mobile" (default), "fixed" or "fax"
\phone[fixed]{+2~(345)~678~901}
\phone[fax]{+3~(456)~789~012}
\email{john@doe.org}                               % optional, remove / comment the line if not wanted
\homepage{www.johndoe.com}                         % optional, remove / comment the line if not wanted
\social[linkedin]{john.doe}                        % optional, remove / comment the line if not wanted
\social[twitter]{jdoe}                             % optional, remove / comment the line if not wanted
\social[github]{jdoe}                              % optional, remove / comment the line if not wanted
\extrainfo{additional information}                 % optional, remove / comment the line if not wanted
\photo[64pt][0.4pt]{picture}                       % optional, remove / comment the line if not wanted; '64pt' is the height the picture must be resized to, 0.4pt is the thickness of the frame around it (put it to 0pt for no frame) and 'picture' is the name of the picture file
\quote{Some quote}                                 % optional, remove / comment the line if not wanted

% bibliography adjustements (only useful if you make citations in your resume, or print a list of publications using BibTeX)
%   to show numerical labels in the bibliography (default is to show no labels)
\makeatletter\renewcommand*{\bibliographyitemlabel}{\@biblabel{\arabic{enumiv}}}\makeatother
%   to redefine the bibliography heading string ("Publications")
%\renewcommand{\refname}{Articles}

% bibliography with mutiple entries
%\usepackage{multibib}
%\newcites{book,misc}{{Books},{Others}}
%----------------------------------------------------------------------------------
%            content
%----------------------------------------------------------------------------------
\begin{document}
%\begin{CJK*}{UTF8}{gbsn}                          % to typeset your resume in Chinese using CJK
%-----       resume       ---------------------------------------------------------
\makecvtitle

\section{Software Skills}
  \cvitem{Javascript}{
    \href{https://github.com/inakianduaga/node-express-boilerplate}{\textbf{Node.js}},
    \href{https://github.com/inakianduaga/github-ci-tasks/tree/develop}{\textbf{TypeScript}} (+ES6/ES7),
    \href{https://github.com/inakianduaga/angular-remote-logger}{\textbf{AngularJS}},
    jQuery,
    Npm,
    \href{http://bower.io/search/?q=angular-remote-logger}{Bower},
    \href{https://coveralls.io/github/inakianduaga/angular-remote-logger}{\textbf{Jasmine}},
    \href{https://github.com/inakianduaga/node-express-boilerplate/blob/master/src/spec/routes/exampleMochaSpec.ts}{Mocha},
    \href{https://github.com/inakianduaga/log-proxy-server/tree/develop/gulp}{\textbf{Gulp}},
    Grunt,
    Webpack.
  }

  \cvitem{PHP}{
    \href{https://github.com/inakianduaga/laravel-html-builder-extensions}{\textbf{Laravel}},
    \href{https://packagist.org/users/inakianduaga/packages/}{\textbf{Composer}},
    \textbf{PHP OOP},
    \href{https://github.com/inakianduaga/eloquent-external-storage/tree/master/tests}{Mockery/Phpunit},
    ORM,
    Types,
    PHPDoc.
  }

  \cvitem{DevOps}{
    \textbf{AWS Stack}: RDS, Cloudfront, EC2, ELB, S3, Route53, OpsWorks, SDK (CLI, NodeJS).
    \href{https://hub.docker.com/r/inakianduaga/}{\textbf{Docker}},
    CI (\href{https://travis-ci.org/inakianduaga/}{\textbf{Travis}}, \href{https://github.com/inakianduaga/docker-jenkins-nginx}{\textbf{Jenkins}}),
    \href{https://github.com/inakianduaga/cloud-development}{\textbf{Nginx}},
    Apache,
    Varnish,
    HHVM,
    Rsyslog,
    Kibana,
    ElasticSearch,
    automatic blue-green deployments,
    \href{https://github.com/inakianduaga/cloud-development/tree/master/scripts/upstart}{Upstart}.
  }

  \cvitem{+Languages}{
    \href{https://github.com/inakianduaga/cloud-development/search?l=bash}{Bash},
    Less,
    HTML5,
    Bootstrap,
    \href{https://gist.github.com/inakianduaga/ef2220f975d8fe6aa88d}{Perl} (tiny bit),
    \href{http://arxiv.org/pdf/0803.3735.pdf}{Latex}.
  }

  \cvitem{Versioning}{
    Git (\href{http://nvie.com/posts/a-successful-git-branching-model/}{git flow}),
    \href{https://github.com/inakianduaga/}{GitHub},
    Bitbucket,
    automatic changelogs,
    tagged releases.
  }

  \cvitem{API}{
    Rest,
    \href{https://www.tado.com/shop/api/public/v1/api-docs}{Swagger},
    OAuth.
  }

  \cvitem{Learning, Queued}{
    \href{https://github.com/inakianduaga/react-redux-typescript/tree/develop}{\textbf{ReactJS (w/ redux pattern)}},
    \textbf{ReactiveX},
    \textbf{AngularJS 2},
    functional programming,
    Scala (\textbf{strong-typed languages}),
    AWS Lambda,
    Hacklang,
    Graph DB,
    WebSockets,
    React Native,
    InfluxDB,
    Grafana.
  }

\section{Experience}

  \subsection{Professional}

    % tado
    \cventry{2013 - Now}{Full Stack Developer}{\href{https://www.tado.com}{tado gmbh}}{Munich}{}{\newline{}%
      In charge of developing \& maintaining the frontend \& backend of the main tado° website, plus an internal dashboard/order management system used by customer support to handle the online shop. In addition, I manage the entire website content lifecycle, including CI infrastructure, AWS EC2 server setup/provisioning, database migrations/backups, code deployments, logging, performance tuning, etc.\newline{}%
      Some of the tasks I frequently do:%
      \begin{itemize}%
        \item Adding functionality and improving the AngularJS webshop.
        \item Implementing advanced backend features to support the complex website structure (DHL/Payone SDKs, Multilanguage support, AWS SNS emails, user accounts, etc).
          \begin{itemize}%
          \item Sub-achievement (a);
          \end{itemize}
        \item Management of the entire deployment pipeline (we deploy continuously, several times a day).
      \end{itemize}}

    % thecouponbay
    \cventry{2010 - 2013}{Admin/Site Owner}{\href{https://web.archive.org/web/20141102092727/http:/thecouponbay.com/}{TheCouponbay.com}}{USA/Munich}{}{\newline{}%
      Site owner and administrator, with the following core functions:
      \begin{itemize}%
        \item Full-stack web developer for all backend and frontend operations sitewide, including entire website design, server security, 3rd party API integrations
        \item Manage marketing operations by creating and running Google Adwords \& Microsoft Bing Ads advertising campaigns through a self-built custom interface.
        \item Establish affiliate partner relationships with publishers such as HP, Dell, Lenovo, Sony and Amazon.com.
      \end{itemize}}
    }

  \subsection{Academic}
    % UIUC
    \cventry{2004-2010}{Teaching Assistant}{University of Illinois at Urbana-Champaign}{Urbana-Champaign}{}{
      I taught physics to undergrads while doing my Ph.D thesis
      \begin{itemize}
        \item Classroom teaching of beginner and intermediate courses in Physics, such as Thermodynamics, Electromagnetism.
        \item Grading of labs, exams and homework assignments.
        \item Ranked as “Excellent Teacher Assistant” on multiple semesters (\href{http://cte.illinois.edu/teacheval/ices/pdf/Sp08List.pdf}{Spring 2008}, \href{http://cte.illinois.edu/teacheval/ices/pdf/Spring09List.pdf}{2009}, \& \href{http://cte.illinois.edu/teacheval/ices/pdf/Spring10List.pdf}{2010}, \href{http://cte.illinois.edu/teacheval/ices/pdf/Fall09List.pdf}{Fall 2009}).
      \end{itemize}
    }


\section{Higher Education}
  \cventry{2004-2010}{Physics Ph.D}{UIUC}{Urbana-Champaign, USA}{\textit{GPA: 3.92}}{
    \begin{itemize}
      \item Specialization in theoretical Condensed Matter Physics
      \item Thesis: \href{https://www.ideals.illinois.edu/handle/2142/18225}{“Many body topics in condensed matter physics”}.
    \end{itemize}
  }
  \cventry{1998-2004}{Physics Bacherlor}{Instituto Balseiro / UNLP}{Argentina}{\textit{8.96/10}}{}
  \cventry{1997}{High School Degree}{Lewiston High School}{Lewiston, USA}{\textit{GPA: 3.9}}{}

\section{Languages}
  \cvitemwithcomment{English}{Fluent}{C2 level, lived 7 years in the USA}
  \cvitemwithcomment{German}{Intermediate}{}
  \cvitemwithcomment{Spanish}{Native speaker}{}

\section{Interests}
  \cvitem{hobby 1}{Description}
  \cvitem{hobby 2}{Description}
  \cvitem{hobby 3}{Description}

\section{Extra 1}
\cvlistitem{Item 1}
\cvlistitem{Item 2}
\cvlistitem{Item 3. This item is particularly long and therefore normally spans over several lines. Did you notice the indentation when the line wraps?}

\section{Extra 2}
\cvlistdoubleitem{Item 1}{Item 4}
\cvlistdoubleitem{Item 2}{Item 5\cite{book1}}
\cvlistdoubleitem{Item 3}{Item 6. Like item 3 in the single column list before, this item is particularly long to wrap over several lines.}

\section{References}
\begin{cvcolumns}
  \cvcolumn{Category 1}{\begin{itemize}\item Person 1\item Person 2\item Person 3\end{itemize}}
  \cvcolumn{Category 2}{Amongst others:\begin{itemize}\item Person 1, and\item Person 2\end{itemize}(more upon request)}
  \cvcolumn[0.5]{All the rest \& some more}{\textit{That} person, and \textbf{those} also (all available upon request).}
\end{cvcolumns}

% Publications from a BibTeX file without multibib
%  for numerical labels: \renewcommand{\bibliographyitemlabel}{\@biblabel{\arabic{enumiv}}}% CONSIDER MERGING WITH PREAMBLE PART
%  to redefine the heading string ("Publications"): \renewcommand{\refname}{Articles}
\nocite{*}
\bibliographystyle{plain}
\bibliography{publications}                        % 'publications' is the name of a BibTeX file

% Publications from a BibTeX file using the multibib package
\section{Publications}
ADD PUBLICATIONS HERE!

%\nocitebook{book1,book2}
%\bibliographystylebook{plain}
%\bibliographybook{publications}                   % 'publications' is the name of a BibTeX file
%\nocitemisc{misc1,misc2,misc3}
%\bibliographystylemisc{plain}
%\bibliographymisc{publications}                   % 'publications' is the name of a BibTeX file

\end{document}


%% end of file `template.tex'.

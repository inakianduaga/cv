%% start of file `template.tex'.
%% Copyright 2006-2015 Xavier Danaux (xdanaux@gmail.com).
%
% This work may be distributed and/or modified under the
% conditions of the LaTeX Project Public License version 1.3c,
% available at http://www.latex-project.org/lppl/.


\documentclass[11pt,a4paper,sans]{moderncv}        % possible options include font size ('10pt', '11pt' and '12pt'), paper size ('a4paper', 'letterpaper', 'a5paper', 'legalpaper', 'executivepaper' and 'landscape') and font family ('sans' and 'roman')

% moderncv themes
\moderncvstyle{casual}                             % style options are 'casual' (default), 'classic', 'banking', 'oldstyle' and 'fancy'
\moderncvcolor{blue}                               % color options 'black', 'blue' (default), 'burgundy', 'green', 'grey', 'orange', 'purple' and 'red'
%\renewcommand{\familydefault}{\sfdefault}         % to set the default font; use '\sfdefault' for the default sans serif font, '\rmdefault' for the default roman one, or any tex font name
% \nopagenumbers{}                                  % uncomment to suppress automatic page numbering for CVs longer than one page

% character encoding
\usepackage[T1]{fontenc}
\usepackage[utf8]{inputenc}                       % if you are not using xelatex ou lualatex, replace by the encoding you are using
%\usepackage{CJKutf8}                              % if you need to use CJK to typeset your resume in Chinese, Japanese or Korean

% http://tex.stackexchange.com/questions/27258/how-do-i-write-underline-text-but-with-a-dotted-line
%\usepackage{ulem} % Enable for dotted text
\usepackage{tikz}

\newcommand{\udot}[1]{%
    \tikz[baseline=(todotted.base)]{
        \node[inner sep=1pt,outer sep=0pt] (todotted) {#1};
        \draw[dotted] (todotted.south west) -- (todotted.south east);
    }%
}%


% adjust the page margins
\usepackage[scale=0.75]{geometry}
% \setlength{\hintscolumnwidth}{2.5cm}                % if you want to change the width of the column with the dates
%\setlength{\makecvtitlenamewidth}{10cm}           % for the 'classic' style, if you want to force the width allocated to your name and avoid line breaks. be careful though, the length is normally calculated to avoid any overlap with your personal info; use this at your own typographical risks...

% personal data
\name{Inaki}{Anduaga}
\title{Curriculum Vitae}                               % optional, remove / comment the line if not wanted
\address{Offenbachstr. 5d}{81245 Munich}{Germany}   % optional, remove / comment the line if not wanted; the "postcode city" and "country" arguments can be omitted or provided empty
%\phone[mobile]{+49~015204256744}                   % optional, remove / comment the line if not wanted; the optional "type" of the phone can be "mobile" (default), "fixed" or "fax"
%\phone[fixed]{+49~8978067867}
\social[github]{inakianduaga}                              % optional, remove / comment the line if not wanted
\email{inaki@inakianduaga.com}                               % optional, remove / comment the line if not wanted
% \homepage{inakianduaga.com}                         % optional, remove / comment the line if not wanted
\social[linkedin]{inakianduaga}                        % optional, remove / comment the line if not wanted
%\social[twitter]{jdoe}                             % optional, remove / comment the line if not wanted
% \extrainfo{additional information}                 % optional, remove / comment the line if not wanted
\photo[64pt][0]{picture}                       % optional, remove / comment the line if not wanted; '64pt' is the height the picture must be resized to, 0.4pt is the thickness of the frame around it (put it to 0pt for no frame) and 'picture' is the name of the picture file
%\quote{Some quote}                                 % optional, remove / comment the line if not wanted

% bibliography adjustements (only useful if you make citations in your resume, or print a list of publications using BibTeX)
%   to show numerical labels in the bibliography (default is to show no labels)
% \makeatletter\renewcommand*{\bibliographyitemlabel}{\@biblabel{\arabic{enumiv}}}\makeatother
%   to redefine the bibliography heading string ("Publications")
%\renewcommand{\refname}{Articles}

% bibliography with mutiple entries
%\usepackage{multibib}
%\newcites{book,misc}{{Books},{Others}}
%----------------------------------------------------------------------------------
%            content
%----------------------------------------------------------------------------------
\begin{document}
%\begin{CJK*}{UTF8}{gbsn}                          % to typeset your resume in Chinese using CJK
%-----       resume       ---------------------------------------------------------
\makecvtitle

\section{About me}

    \textbf{Code must get better. Every day. Every time. Every line.} A relentless passion for better code is what drives
     me late at night to read about new technologies, new tools, and new languages, and has brought me a long way in a short few years.
     But I'm nowhere near where I want to be. In software engineering, with its ever-changing languages and technologies, there is always something to be excited about,
     with fun and incredible things awaiting to be built. If feel there is so much to learn, yet so little time\ldots \newline{}

    % Like a young child craving for new Lego pieces, programming, with its ever-changing languages and technologies, brings endless fun and new opportunities to improve.
        % Part of what makes programming so much fun is that it is an ever-changing landscape, with an opportunity to innovate and learn around the corner.

    I've been solving problems for the past fifteen years, first as a physicist fascinated by the mathematical beauty and complexity of nature's laws, and nowadays as a software
    engineer working on more practical challenges, always aspiring for the the most efficient and clean solution. Although different problems require different tools, the underlying thinking
    is usually the same, namely consistant logic, big-picture analysis, generalization of specific solutions, and looking at edge cases, whether working with a
    differential equation or writing a unit test. These are the concepts I work hard to become better at every day.

\nopagebreak[4]

\section{Software Skills}
  \cvitem{Scala}{
    \href{https://github.com/inakianduaga/kafka-image-processor/tree/master/ui-backend}{\textbf{Play}},
    Akka Actors, Sangria GraphQL, Type Classes, RxScala, Circe    
  }

  \cvitem{Javascript}{
    \href{https://github.com/inakianduaga/react-native-demo}{\textbf{React Native}} \href{https://inakianduaga.github.io/react-native-demo/#1}{\textbf{(talk)}},  
    \href{https://github.com/inakianduaga/react-redux-typescript/tree/develop}{\textbf{React}} \href{https://github.com/inakianduaga/redux-state-history}{\textbf{(Redux)}},
    \href{https://github.com/inakianduaga/github-ci-tasks/tree/develop}{\textbf{TypeScript}},
    \href{https://github.com/inakianduaga/kafka-image-processor/tree/master/ui/src}{\textbf{CycleJS}},
    \href{https://github.com/inakianduaga/angular-remote-logger}{\textbf{AngularJS}},
    RxJS,
    \href{https://github.com/inakianduaga/node-express-boilerplate}{\textbf{NodeJS}},
    \href{https://github.com/inakianduaga/babel-plugin-markdown-compiler/tree/develop}{\textbf{Babel}},
    Jest,
    \href{https://coveralls.io/github/inakianduaga/angular-remote-logger}{\textbf{Jasmine}},
    \href{https://github.com/inakianduaga/node-express-boilerplate/blob/master/src/spec/routes/exampleMochaSpec.ts}{\textbf{Mocha}},
    \href{https://www.npmjs.com/~inakianduaga}{\textbf{Npm}},
    Webpack,
    \href{https://github.com/inakianduaga/log-proxy-server/tree/develop/gulp}{\textbf{Gulp}}
  }

  \cvitem{DevOps}{
    AWS: Cloudformation, Lambda, Kinesis, ECS, EC2, RDS, SDK, Route53 and more\newline
    Logging/Metrics: JMX, Datadog, OpsGenie, Rsyslog, Kibana, ElasticSearch\newline
    CI: \href{https://travis-ci.org/inakianduaga/}{\textbf{Travis}}, \href{https://github.com/inakianduaga/docker-jenkins-nginx}{\textbf{Jenkins}}, GoCD\newline
    Containers: \href{https://hub.docker.com/r/inakianduaga/}{\textbf{Docker}}, \href{https://github.com/inakianduaga/kafka-image-processor/blob/master/docker/docker-compose.yml}{\textbf{Docker Compose}}\newline
    Event Sourcing: \href{https://github.com/inakianduaga/kafka-image-processor}{\textbf{Kafka / Avro}}, \href{https://github.com/inakianduaga/kafka-image-processor/blob/master/processor/src/main/scala/com/inakianduaga/Kafka.scala#L29}{\textbf{Akka Kafka Streams}}, ReactiveX\newline
    Servers: \href{https://github.com/inakianduaga/cloud-development}{\textbf{Nginx}}, Apache, Varnish, Ansible
  }

  \cvitem{PHP}{
    \href{https://github.com/inakianduaga/laravel-html-builder-extensions}{\textbf{Laravel}},
    \href{https://packagist.org/users/inakianduaga/packages/}{\textbf{Composer}},
    \href{https://github.com/inakianduaga/eloquent-external-storage/tree/master/tests}{\textbf{Mockery/PHPUnit}}
  }

  \cvitem{Other}{
    \textbf{\href{https://github.com/inakianduaga/cloud-development/search?l=bash}{Bash}},
    \textbf{\href{http://arxiv.org/pdf/0803.3735.pdf}{Latex}},
    Code quality tools (Scalafmt, PrettierJS, TSLint)
  }

  % \cvitem{Versioning}{
  %   Git (\href{http://nvie.com/posts/a-successful-git-branching-model/}{git flow}),
  %   \href{https://github.com/inakianduaga/}{GitHub},
  %   Bitbucket,
  %   Automatic Changelogs,
  %   Tagged Releases.
  % }
  \cvitem{API}{
    GraphQL, Apollo Client,
    Rest, \textbf{\href{https://github.com/inakianduaga/scala-play-car-advert/blob/master/public/swagger/carAdverts.yml}{Swagger}}
  }

  \cvitem{Learning...}{
    Machine learning,  
    Python,
    Functional Programming,
    Kubernetes,
    Scala Shapeless
  }

  \cvitem{}{}
  \cvitem{}{
    Pro Tip: Click on the bolded skills to view examples of my work.
  }

\section{Professional Experience}

  %\subsection{Professional}

    % autoscout24
    \cventry{2016 - present}{Lead Engineer}{\href{https://www.autoscout24.com}{AutoScout24 gmbh}}{Munich, Germany}{}{\\
      In charge of pushing innovation at autoscout24 through research of new technologies \/ frameworks, introduction of best coding practices and improving overall technical abilities of fellow engineers. 
      %Since my start at the beginning of 2017, I've worked towards AutoScout24 goal of merging the mobile and web teams by introducing my team to React / TypeScript, while self-learning and then teaching/advocating for React Native and GraphQL. 
      %This has put our team in an excellent technical position to complete the integration in early 2018, and spread the knowledge to other teams. 
      
      %Some of the things I've been working on and achieved so far:%
      % \begin{itemize}%
      %   \item Push for React Native as the solution to integrate mobile development into existing traditional \"web\" teams (ongoing)
      %   \item Push \& use GraphQL for newer APIs, providing clients with state-of-the-art frontend tooling
      %   \item Introduce modern React / TypeScript / ImmutableJS / RxJS frontend stack and migrate existing legacy UI codebase (ongoing)
      %   \item Introduce linting/formatting tools for easier team collaboration and more consistent code.
      %   \item Introduce PR-based commit approach where every contribution is peer-reviewed, raising code awareness and quality through feedback.
      %   \item Mentor Junior \& new engineers in both company processes and programming know-how.  
      % \end{itemize}}

    % tado
    \cventry{2013 - 2015}{Full Stack Developer}{\href{https://www.tado.com}{tado gmbh}}{Munich, Germany}{}{\\
      In charge of developing and maintaining the entire frontend \& backend of the main \href{https://www.tado.com}{tado.com} website, including CI infrastructure, AWS EC2 server setup/provisioning, database migrations/backups, code deployments, logging \& performance tuning \newline{}%
      Some of the responsabilities included:%
      \begin{itemize}%
        \item Upgrading code & content for the tado.com website and its ReactJS-based webshop.
        \item Management of the entire deployment pipeline (incl. daily continuous deployments).
        \item Implementing advanced backend features to support the complex website structure (external SDKs, multilanguage support, order invoicing, user accounts).
      \end{itemize}}

    % thecouponbay
    \cventry{2010 - 2013}{Admin/Site Owner}{\href{https://web.archive.org/web/20141102092727/http:/thecouponbay.com/}{TheCouponbay.com}}{USA/Munich}{}{\newline{}%
      Site owner and administrator, with the following core functions:
      \begin{itemize}%
        \item Full-stack web developer for all backend and frontend operations sitewide, including entire website design, server security, 3rd party API integrations
        \item Manage marketing operations by creating and running Google Adwords \& Microsoft Bing Ads advertising campaigns through a self-built custom interface.
        \item Establish affiliate partner relationships with publishers such as HP, Dell, Lenovo and Sony.
      \end{itemize}}


  %\subsection{Academic}
    % UIUC
    \cventry{2004-2010}{Teaching Assistant}{UIUC}{Urbana-Champaign, USA}{}{
      \begin{itemize}
        \item Classroom teaching of beginner and intermediate courses in Physics, such as Thermodynamics, Electromagnetism.
        % \item Grading of labs, exams and homework assignments.
        \item Ranked as “Excellent Teacher Assistant” on multiple semesters (\href{http://cte.illinois.edu/teacheval/ices/pdf/Sp08List.pdf}{2008}, \href{http://cte.illinois.edu/teacheval/ices/pdf/Spring09List.pdf}{2009}, \& \href{http://cte.illinois.edu/teacheval/ices/pdf/Spring10List.pdf}{2010}).
      \end{itemize}
    }


\section{Higher Education}
  \cventry{2004-2010}{Physics Ph.D}{UIUC}{Urbana-Champaign, USA}{\textit{GPA: 3.92}}{
    \begin{itemize}
      \item Specialization in theoretical Condensed Matter Physics
      \item Thesis: \href{https://www.ideals.illinois.edu/handle/2142/18225}{“Many body topics in condensed matter physics”}.
    \end{itemize}
  }
  \cventry{1998-2004}{Physics Bacherlor}{Instituto Balseiro / UNLP}{Argentina}{\textit{8.96/10}}{}

\section{Languages}
  \cvitemwithcomment{English}{Fluent}{C2 level, lived 7 years in the USA}
  \cvitemwithcomment{German}{Intermediate}{}
  \cvitemwithcomment{Spanish}{Native speaker}{}

\section{Publications}

  \cventry{2008}{Michael Stone, Inaki Anduaga and Lei Xing}{}{The classical hydrodynamics of the Calogero–Sutherland model}{\textit{J. Phys. A: Math. Theor. 41 275401 doi:10.1088/1751-8113/41/27/275401}}{}
  \cventry{2007}{Michael Stone, Inaki Anduaga}{}{Mass flows and angular momentum density for $ p_x + i p_y $ paired fermions in a harmonic trap}{\textit{Annals of Physics DOI:10.1016/j.aop.2007.04.020}}{}

\end{document}

%% end of file `template.tex'.
